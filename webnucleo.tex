\documentclass{article}
\setlength\parindent{0pt}
\title{An Introduction to Webnucleo:\\
        A Webgate for the Masses}

\author{Jaad A. Tannous, Bradley S. Meyer\\
        Clemson University}
\date{}

\begin{document}
\maketitle
\begin{abstract}
In this paper, we introduce Webnucleo.org
\end{abstract}

\section*{Introduction}
Numerical tools have been the bread and butter for conceptual understanding and order of magnitude 
estimates for scientists since the 50s. With advancements in processing power, memory management, and 
internet access, numerical tools and programming languages proliferated the scientific community. Each 
subfield in scientific community has a niche set of algorithms and codes to work with. With accessibility
and ease of use in mind, the Webnucleo team, based in the department of physics and astronomy at Clemson university, has built 
a webgate of codes that focus on applications in astrophysics that can be found at www.webnucleo.org. Specifically, our applications
focus on nuclear astrophysics, nucleosynthesis, stellar evolution, astronomy, and cosmochemistry. We are producing a number
of products that researchers, educators, and students may find useful in studying the aforementioned topics of interest. 
These products include open-source codes, Docker images, and Jupyter notebooks.\\

To be a true webgate for the masses, we house the codes themselves in GitHub repositories controlled 
by the authors and are published under GPL-3.0 licenses. They are open source and contributions to the codes can be 
done through GitHub. Webnucleo.org itself is hosted on Read The Docs, a platform which is easy to navigate. Interested 
groups may contribute any developed tools in the field to the webgate by simply submitting  a ticket through webnucleo.org.\\

The codes added to the webgate are all self-contained within their respective environments. Simply put, there is no 
elaborate package and library management required. Should the tool you are looking at requires any particular package, either 
explicit documentation is provided or the tool will automatically handle the installation. The remainder 
of the paper highlights the Jupyter notebooks currently available, which are categorized as nucleosynthesis, stellar structure, 
and webnucleo XML building.

\section*{Nucleosynthesis}

Since its postulation by Eddington in 1920, nucleosynthesis has expanded beyond simple nuclear fusion to include a plethora of 
capture and decay processes, whose prevalence is determined by the thermodynamic properties of the environment. The Webnucleo team 
is primarily a thermonuclear astrophysics group, so naturally understanding mainline stellar nucleosynthesis is foundational to 
understand energy production in stars. The first notebook describes precisely that. It utilizes a simplistic temperature and density 
model to help describe core nuclear fusion in stars. It illustrates core burning in all mass ranges, covering all the way from 
Hydrogen to Silicon burning.\\

Nuclear fusion however, is insufficient to create nuclei heavier than iron. For that, other capture processes are required. Within 
the lifetime of the star, one such process occurs. The slow neutron capture process, s-process, occurs during thermal pulsations in 
the asymptotic giant branch of low and intermediate mass stars, and in the shells of massive stars beyond Hydrogen burning. 
The process itself is slow since in the time between neutron captures, $\beta$- decay may occur, creating heavier species. One notebook 
explores the s-process for a constant neutron number density, while the other helps build an intuition on the back and forth transitions 
between the ground and isomeric state of isotopes. The latter is important since certain isotopes, such as $^{85}Kr$, act as branching 
points for the s-process. A third explores NRLEE (Neutron-Rich, Low-Entropy matter Ejector) nucleosythesis in simple models.\\

The other neutron capture process is the rapid neutron capture process; r-process. It occurs in high temperature and neutron number 
density environments such as in nova shells. It is rapid because a given isotope can capture many neutrons before it $\beta$- decays. 
A bundle of notebooks were developed to study the various equilibria that occur during the r-process.

\section*{Stellar Structure}

Understanding the environment in which nucleosynthesis occurs helps put things in context. Stellar modeling in and of itself is quite 
an intricate undertaking. Building a code that solves the coupled structure equations, determines the appropriate opacity, calculates the thermodynamics, 
and handles the nuclear network is quite the challenge. Two notebooks have been incorporated to help set users on the right path.\\

The first of the two handles profiling a star's interior through the Polytropic equation of state. Polytropes with the appropriate index
can model stellar interiors and white dwarfs quite well. The notebook available explains the derivation of the Lane-Emden equation, then 
solves it, and builds a mass, temperature, and pressure profiles for main sequence stars and white dwarfs. One may experiment with different 
polytropic indeces, central temperatures and densities and get corresponding profiles with a few keystrokes and a click of a button.\\ 

The second notebook alleviates the mystique behind solving the coupled stellar structure equations. The established method of solving 
the system is by way of the Henyey method. The method has the Newton-Raphson root solving technique for coupled equations with a 
relaxation factor. The notebook plunges through the derivation of simple case of two coupled partial differential equations, with an example. 
The physics related example illustrated is the time evolution of mass distribution of a galaxy. A system of 3 coupled first order PDEs 
and the initial condition has all the mass in its halo. The method is easily extendable to any square system and conserves the appropriate 
quantities. The only issues arise is trying to invert matrices large matrices. 

\section*{Webnucleo XML}

To study different effects on nucleosynthesis, the webnucleo team uses libnucnet. It is an in-house network developed by Prof. Bradley 
S. Meyer over the years. The network itself uses the JINA reaction rates and can run s and r-process calculations after setting the 
appropriate initial conditions. The code itself utilizes XML format for data structuring and allows users to vary rates and abundances 
easily by simply using their own XML files as input for the network.\\

To create such XML files, a series of notebooks have been written to easily illustrate their construction, and another to visualize 
the newly created XMLs.

\section*{Final Remarks}

At the time of writing, the webgate seems rather limited. Notebooks and other packages are constantly being polished and added to the 
webnucleo.org. As mentioned, the use for these packages are not strictly for research, but can also be utilized in the classroom as 
visual aids, or for students and enthusiasts alike to test and tweak parameters in a simple environment that does require much 
experience in programming. This accessibility allows anyone with sufficient background to understand these involved subjects and have 
read-to-use tools at their finger tips. 

\end{document}
