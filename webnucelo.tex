\documentclass{article}
\setlength\parindent{0pt}
\title{An Introduction to Webnucleo:\\
        A Webgate for the Masses}

\author{Jaad A. Tannous, Bradley S. Meyer\\
        Clemson University}
\date{}

\begin{document}
\maketitle
\begin{abstract}
In this paper, we introduce Webnucleo.org
\end{abstract}

\section*{Introduction}
Numerical tools have been the bread and butter for conceptual understanding and order of magnitude 
estimates for scientists since the 50s. With advancements in processing power, memory management, and 
internet access, numerical tools and programming languages proliferated the scientific community. Each 
subfield in scientific community has a niche set of algorithms and codes to work with. With accessibility
and ease of use in mind, the Webnucleo team, based in the department of physics and astronomy at Clemson university, has built 
a webgate of codes that focus on applications in astrophysics that can be found at www.webnucleo.org. Specifically, our applications
focus on nuclear astrophysics, nucleosynthesis, stellar evolution, astronomy, and cosmochemistry. We are producing a number
of products that researchers, educators, and students may find useful in studying the aforementioned topics of interest. 
These products include open-source codes, Docker images, and Jupyter notebooks.\\

To be a true webgate for the masses, we house the codes themselves in GitHub repositories controlled 
by the authors and are published under GPL-3.0 licenses. They are open source and contributions to the codes can be 
done through GitHub. Webnucleo.org itself is hosted on Read The Docs, a platform which is easy to navigate. Interested 
groups may contribute any developed tools in the field to the webgate by simply submitting  a ticket through webnucleo.org.\\

The codes added to the webgate are all self-contained within their respective environments. Simply put, there is no 
elaborate package and library management required. Should the tool you are looking at requires any particular package, either 
explicit documentation is provided or the tool will automatically handle the installation. The remainder 
of the paper highlights the Jupyter notebooks currently available, which are categorized as nucleosynthesis, stellar structure, 
and webnucleo XML building.

\section*{Nucleosynthesis}

Since its postulation by Eddington in 1920, nucleosynthesis has expanded beyond simple nuclear fusion to include a plethora of 
capture and decay processes, whose prevalence is determined by the thermodynamic properties of the environment. The Webnucleo team 
is primarily a thermonuclear astrophysics group, so naturally understanding mainline stellar nucleosynthesis is foundational to 
understand energy production in stars. The first notebook describes precisely that. It utilizes a simplistic temperature and density 
model to help describe core nuclear fusion in stars. It illustrates core burning in all mass ranges, covering all the way from 
Hydrogen to Silicon burning.\\

Nuclear fusion however, is insufficient to create nuclei heavier than iron. For that, other capture processes are required. Within 
the lifetime of the star, one such process occurs. The slow neutron capture process, s-process, occurs during thermal pulsations in 
the asymptotic giant branch of low and intermediate mass stars, and in the shells of massive stars beyond Hydrogen burning. 
The process itself 


\section*{Stellar Structure}

\section*{Webnucleo XML}


\end{document}